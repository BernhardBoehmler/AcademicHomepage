\documentclass[10pt,a4paper,sans]{moderncv}
\usepackage{graphicx}
\usepackage{lastpage}
\rfoot{\addressfont\itshape\textcolor{gray}{Page \thepage\ of \pageref{LastPage}}}
\moderncvstyle{casual}
\moderncvcolor{blue}

% Spaltenbreite ändern 
\setlength{\hintscolumnwidth}{3.3cm} %Breite linke Spalte 
\setlength{\separatorcolumnwidth}{0.5cm} %Breite mittlere Spalte 
\setlength{\maincolumnwidth}{12.5cm} %Breite rechte Spalte 


\usepackage{comment}
\usepackage{enumitem}
\usepackage[scale=0.75]{geometry}
\geometry{left=1.7cm, right=1.7cm, top=0.5cm, bottom=0.5cm, includeheadfoot, footskip=39pt,headheight=13.6pt,headsep=10pt} 


\pagestyle{fancy}
\fancyhf{}
\fancyhead[L]{\addressfont\itshape\textcolor{gray}{Curriculum vitae}}\fancyhead[C]{}\fancyhead[R]{\addressfont\itshape\textcolor{gray}{Bernhard B\"ohmler}}
\fancyfoot[L]{}\fancyfoot[C]{Centre footer}\fancyfoot[R]{\addressfont\itshape\textcolor{gray}{Page \thepage\ of \pageref{LastPage}}}
\fancypagestyle{firstpage}{%
	\lhead{}
	\chead{}
	\rhead{}
}




%\usepackage[nottoc]{tocbibind}

\usepackage{enumitem}
\usepackage{amsmath}

\usepackage[english]{babel}

%\nopagenumbers

\newcommand*{\httpslink}[2][]{%
	\ifthenelse{\equal{#1}{}}%
	{\href{https://#2}{#2}}%
	{\href{https://#2}{#1}}}
\firstname{Bernhard}
\familyname{B\"ohmler}
\title{ - Curriculum vitae - }


% personal data
% optional, remove / comment the line if not wanted
\address{Kurt-Schumacher-Stra{\ss}e 6}{67663 Kaiserlslautern}{Germany}% optional, remove / comment the line if not wanted; the "postcode city" and "country" arguments can be omitted or provided empty
%\phone[mobile]{+49~(0)~176~846~624~94}              % optional, remove / comment the line if not wanted; the optional "type" of the phone can be "mobile" (default), "fixed" or "fax"
%\phone[fixed]{+49~(0)~631~205~4789}
\email{boehmler@mathematik.uni-kl.de}                               % optional, remove / comment the line if not wanted
%\homepage{www.johndoe.com}                         % optional, remove / comment the line if not wanted
% optional, remove / comment the line if not wanted
%\extrainfo{additional information}                 % optional, remove / comment the line if not wanted
% optional, remove / comment the line if not wanted; '64pt' is the height the picture must be resized to, 0.4pt is the thickness of the frame around it (put it to 0pt for no frame) and 'picture' is the name of the picture file
%\quote{Some quote} 





%\extrainfo{12334567890}
%%%%%%%%%%%%%%%%%%%%%%%%%%%%%%%%%%%%%%%%%%%%%%%%%%%%%%%%%%%%%%%%%%%%%%%%%%%%%%
%% %%
%%%%%%%%%%%%%%%%%%%%%%%%%%%%%%%%%%%%%%%%%%%%%%%%%%%%%%%%%%%%%%%%%%%%%%%%%%%%%%



\begin{document}
	\makecvtitle
	\thispagestyle{firstpage}
	\leavevmode
	\vspace{-\baselineskip}
	\vspace{-\baselineskip}
	\vspace{-\baselineskip}
	\vspace{-\baselineskip}
	\vspace{-17pt}
	\section{Biographical information}
	\cvitem{Place and date of birth}{$28^\text{th}$ July $1988$, Leonberg, Germany}
	\cvitem{Nationality}{German}
	\section{Language skills}
	\cvitem{Mother tongue}{German}
	\cvitem{Foreign languages}{English (level B2), Latin (high school level), Spanish (level A2), French (level A1)}
	\section{Education}
	\cvitem{\mbox{08/2018 - 09/2023}}{\textbf{Doctoral student in mathematics, TU Kaiserslautern} (funded by~the~DFG)}
	\cvitem{ }{Topic: Trivial source character tables of small finite groups\newline Supervisor: Jun.-Prof. Dr. Caroline Lassueur}
	
	\cvitem{\mbox{11/2011 - 09/2016}}{\textbf{Master of Science in mathematics, University of Stuttgart}}
	\leavevmode
	\vspace{-\baselineskip}
	\cvitem{ }{Specialisations:}
	\vspace{-8pt}
	\cvitem{}{\begin{itemize}[noitemsep]
			\item Representation theory of finite-dimensional algebras
			\item Homological algebra
			\item Functional analysis
			\item Theoretical physics (minor subject)
			\item Master's Thesis (04/2016 - 09/2016)\newline Title: Contributions to the representation theory of gendo-symmetric algebras\newline
			Supervisor: Prof. Dr. Wolfgang Rump
	\end{itemize}}
	\cvitem{10/2008 - 11/2011}{\textbf{Bachelor of Science in mathematics, University of Stuttgart}}
	\leavevmode
	\vspace{-\baselineskip}
	\cvitem{ }{Specialisations:}
	\vspace{-8pt}
	\cvitem{}{\begin{itemize}[noitemsep]
			\item Representation theory of finite groups
			\item Algebra
			\item Experimental physics (minor subject)
			\item Bachelor's Thesis (05/2011 - 10/2011)\newline Title: Influence of character degrees on the group structure
	\end{itemize}}
	\cvitem{09/1999 - 07/2008}{\textbf{Abitur, Albert-Schweitzer-Gymnasium Leonberg}}
	\section{Research interests}
	\cvitem{}{- Group theory}
	\cvitem{}{- Modular and ordinary representation theory of finite groups}
	\cvitem{}{- Representation theory of finite-dimensional algebras and related combinatorial structures}
	\cvitem{}{- Auslander-Reiten theory}
	\cvitem{}{- Homological algebra}
\section{Publications $\&$ preprints}
\cvitem{In preparation}{\textit{An algorithm to compute trivial source character tables of finite groups}}
\cvitem{06/2022}{\textit{On the extension-closed property for the subcategory $\textup{Tr}(\Omega^2(\textup{mod} - A))$}\newline\noindent
	{DOI: \url{https://doi.org/10.1007/s10468-022-10140-7}}\newline\noindent
	{Algebr. Represent. Theory} (2022)\ \ \textup{(with Ren{\'e} Marczinzik)}}
\cvitem{05/2022}{\textit{Trivial source character tables of $\textup{SL}_2(q)$}\newline\noindent J. Algebra \textbf{598} (2022), 308-350 \textup{(with Niamh Farrell and Caroline Lassueur)}}
\cvitem{01/2022}{\textit{A cluster tilting module for a representation-infinite block of a group algebra}\newline\noindent J. Algebra \textbf{589} (2022), 483-494 \textup{(with Ren{\'e} Marczinzik)}}
\cvitem{01/2018}{\textit{On a conjecture about Morita algebras}\newline\noindent J. Algebra \textbf{508} (2018), 569-574 (with Ren{\'e} Marczinzik)}
	\section{Research stays}
	\cvitem{10/2021 - 01/2022}{4 months at the \textit{University of California, Santa Cruz}, scholarship granted by the German Academic Exchange Service (DAAD)}
	\cvitem{03/2021}{2 weeks at \textit{Mathematisches Forschungsinstitut Oberwolfach}, Oberwolfach Research Fellowship}
	\cvitem{01/2020}{1 week at the \textit{Isaac Newton Institute for Mathematical Sciences}, University of Cambridge}
	\pagestyle{empty}
	\pagestyle{fancy}
	\fancyhf{}
	\fancyhead[L]{\addressfont\itshape\textcolor{gray}{Curriculum vitae}}\fancyhead[C]{}\fancyhead[R]{\addressfont\itshape\textcolor{gray}{Bernhard B\"ohmler}}
	\fancyfoot[L]{}\fancyfoot[C]{}\fancyfoot[R]{\addressfont\itshape\textcolor{gray}{Page \thepage\ of \pageref{LastPage}}}
	%\section{Seminars organised}
	%\cvitem{10/2021}{SFB-TRR 195 joint block seminar on representation theory of finite groups and quantum groups in %Dagstuhl (hybrid meeting)}
%	\newpage
	\section{Contributed talks}
	\subsection{Research talks}
	\cvitem{09/2022}{{\textbf{On the computation of trivial source character tables using computer algebra}} during the conference \textit{Representation Theory
			at the Villa Denis} in Kaiserslautern}
	\cvitem{08/2022}{{\textbf{On the computation of trivial source character tables using computer algebra}} during the conference \textit{Structure of Group Algebras over Local Rings} in Ambleside (England)}
	\cvitem{09/2021}{{\textbf{On the computation of trivial source character tables}} during the \textit{Young Algebraists' Conference} at the EPFL (Switzerland)}
	\cvitem{02/2021}{{\textbf{Brou{\'e}'s abelian defect group conjecture via $p$-permutation equivalences}} during the \mbox{seminar} \textit{Promotionen und Postdocs in Darstellungstheorie} (online)}
	\cvitem{02/2020}{\textbf{Charaktertafeln von Moduln kleiner endlicher Gruppen mit trivialen Quellen}\newline during the \textit{Gr{\"u}ppchen} at the Martin-Luther-University in Halle/Saale}
	\cvitem{01/2020}{\textbf{Trivial source character tables of small finite groups}\newline during the  \textit{IRTG-seminar} at the Saarland University}
	\cvitem{09/2019}{\textbf{Representation type and combinatorics} at the $83^\text{rd}$ \textit{S{\'e}minaire Lotharingien de Combinatoire} in Bad Boll}
	\subsection{Working group talks}
	\cvitem{06/2021}{{\textbf{The Littlewood-Richardson rule}} during the seminar \textit{Representation Theory of the Symmetric Group} at the TU Kaiserslautern}
	\cvitem{01/2021}{{\textbf{$p$-permutation equivalences}} during the seminar \textit{Equivalences of block algebras} at the TU Kaiserslautern}
	\cvitem{06/2019}{{\textbf{Blocks with cyclic defect groups I}} during the seminar \textit{Block Theory} at the TU Kaiserslautern}
	\cvitem{04/2019}{{\textbf{On a counter-example to a conjecture about Morita algebras}} during the seminar \textit{Groups and Representations} at the TU Kaiserlslautern}
	\cvitem{10/2018}{{\textbf{Projective and injective modules}} during the seminar \textit{Perverse equivalences and applications} at the TU Kaiserslautern}
%	\section{Contributed posters}
%	\cvitem{01/2020}{Poster presented during GRAW01 (Groups, Representations and Applications - introductory workshop) at the Isaac Newton Institute of Mathematical Sciences, University of Cambridge}
%	\cvitem{09/2019}{Poster presented during the annual conference of the SFB-TRR 195 at the Saarland University}
	\section{Selected attended conferences and workshops}
	\cvitem{10/2022}{Darstellungstheorietage in Kaiserslautern}	
	\cvitem{09/2022}{Representation Theory at the Villa Denis in Kaiserslautern}
	\cvitem{08/2022}{Structure of Group Algebras over Local Rings in Ambleside (England)}
	\cvitem{09/2021}{Young Algebraists' Conference at the EPFL (Switzerland)}
	\cvitem{02/2021}{Promotionen und Postdocs in Darstellungstheorie (online)}
	\cvitem{12/2020}{Nikolaus conference (online)}
	\cvitem{09/2020}{SFB-TRR 195: Annual conference (online)}
	\cvitem{02/2020}{{Gr{\"u}ppchen} at the Martin-Luther-University in Halle/Saale}
	\cvitem{12/2019}{Nikolaus conference in Aachen}
	\cvitem{10/2019}{Darstellungstheorietage in Jena}
%	\cvitem{09/2019}{SFB-TRR 195: Annual conference in Saarbr{\"u}cken}
	\cvitem{06/2019}{Norddeutsches Gruppentheorie-Kolloquium in Halle}
	\cvitem{06/2019}{Groups, Rings and Associated Structures in Spa (Belgium)}
	\cvitem{12/2018}{Nikolaus conference in Aachen}
	\cvitem{09/2018}{Darstellungstheorietage in Hannover}
%	\cvitem{09/2018}{SFB-TRR 195: Annual conference in T{\"u}bingen}
	\cvitem{03/2016}{Workshop on Brauer Graph Algebras at the University of Stuttgart (as aide)}
	\cvitem{03/2016}{Conference on Triangulated Categories in Algebra, Geometry and Topology\newline University of Stuttgart (as aide)}
	\cvitem{09/2015}{Third GAP Days 2015 at the NTNU in Trondheim (Norway)}
	\section{University teaching}
	\cvitem{04/2022 - 09/2022}{B.Sc. lecture: Character Theory of Finite Groups, assistant}
	\cvitem{04/2022 - 09/2022}{B.Sc. lecture: Complex Analysis for Engineers, assistant}
	\cvitem{04/2021 - 09/2021}{M.Sc. lecture: Cohomology of Groups, assistant}
	\cvitem{04/2021 - 09/2021}{B.Sc. lecture: Complex Analysis for Engineers, assistant}
	\cvitem{10/2020 - 03/2021}{M.Sc. lecture: Representation Theory of Finite Groups, assistant}
	\cvitem{04/2019 - 09/2019}{B.Sc. lecture: Character Theory of Finite Groups, assistant}
	\cvitem{10/2016 - 03/2017}{B.Sc. lecture: Higher Mathematics 3 for Engineers, tutor}
	\cvitem{10/2015 - 12/2015}{B.Sc. lecture: Higher Mathematics 3 for Electrical Engineering Technicians, tutor}
	\cvitem{10/2014 - 03/2015}{B.Sc. lecture: Higher Mathematics 3 for Engineers, tutor}
	\cvitem{04/2014 - 09/2014}{B.Sc. lecture: Higher Mathematics 2 for Engineers, tutor}
	\cvitem{10/2013 - 03/2014}{B.Sc. lecture: Higher Mathematics 3 for Engineers, tutor}
	\cvitem{04/2013 - 09/2013}{B.Sc. lecture: Algebra for Mathematicians, tutor}
	\cvitem{10/2012 - 03/2013}{B.Sc. lecture: Higher Mathematics 1 for Engineers, tutor}
	\cvitem{04/2012 - 09/2012}{B.Sc. lecture: Geometry for Mathematicians, tutor}
%	\cvitem{10/2011 - 03/2012}{B.Sc. lecture: Higher Mathematics 1 for Engineers, tutor}
	
	
	
	
	
	\section{Work experience}
	\cvitem{09/2017 - 07/2018}{High school teacher for mathematics and physics at the Stetten-Institute in Augsburg}
	\cvitem{05/2017 - 08/2017}{Freelance collaborator at Sch{\"u}lerhilfe Leonberg (tutor for mathematics)}
	\section{Computer skills}
	\cvitem{Typesetting}{\LaTeX{} (very good)}
	\cvitem{Computer languages}{GAP (very good), MAGMA (good), SAGE (basics)}
	\section{Extracurricular activities}
	\subsection{Further training}
	\cvitem{04/2018}{DeltaPlus-course \textit{Mathematik sprachsensibel unterrichten} in Augsburg}
	\cvitem{01/2018}{Course \textit{Die Kunst des guten Unterrichtens} with Dr. Siegfried Rodehau in N{\"u}rnberg}
	\cvitem{11/2017}{Colloquium \textit{Warum Geraden nicht Gerade sein m{\"u}ssen} about synthetic geometry with Prof. Wolfgang Schneider in Augsburg}
	\cvitem{11/2017}{Courses \textit{"Sprachsensibler Unterricht", "Umgang mit Notfallsituationen", "Gespr{\"a}chsf{\"u}hrung"} in the retreat house of St. Paulus in Leitershofen}
	\subsection{Special qualifications}
	\cvitem{10/2011 - 03/2012}{Key qualification course for tutors of Higher Mathematics at the University of Stuttgart}
	\subsection{Voluntary services}
	\cvitem{04/2019 - today}{Clarinet player at the \textit{university orchestra} of the TU Kaiserslautern}
	\cvitem{09/2019 - 03/2020}{Tutor for mathematics at the \textit{learning centre} of the TU Kaiserslautern}
	\cvitem{1999 - today}{Member of (the executive board of) several \textit{chess clubs}}
	\cvitem{1996 - 2016}{Member of the wind band \textit{Musikverein Stadtkapelle Leonberg} and leader of the clarinets}
	\subsection{Awards}
	\cvitem{03/2008}{1$^\text{st}$ price at the federal state music competition \textit{Jugend musiziert} (solo piano)}
	\cvitem{03/2007}{1$^\text{st}$ price at the federal state music competition \textit{Jugend musiziert} (Duo: clarinet \& piano)}
	\cvitem{05/2006}{3$^\text{rd}$ price at the national music competition \textit{Jugend musiziert} (solo clarinet)}
	\cvitem{03/2006}{1$^\text{st}$ price at the federal state music competition \textit{Jugend musiziert} (solo clarinet)}
	% \cvitem{03/2006}{3$^\text{rd}$ price at the regional mathematics competition
	% \textit{Bundeswettbewerb Mathematik}}
	\cvitem{10/2005}{2$^\text{nd}$ price at the \textit{Matthaes} piano competition in Stuttgart (solo piano)}
\section{Referees}
\cvitem{ }{Caroline Lassueur, lassueur@mathematik.uni-kl.de, +49 (0)631 205 2515}
%,\newline \url{https://www.mathematik.uni-kl.de/~lassueur/en/}\newline}
\cvitem{ }{Robert Boltje, boltje@ucsc.edu, +1 (0)831 459 5001}
	%,\newline \url{https://boltje.math.ucsc.edu/}\newline}
\cvitem{ }{Wolfgang Kimmerle, wolfgang.kimmerle@mathematik.uni-stuttgart.de, +49 (0)711 685 65323}
	%,\newline \url{https://www.idsr.uni-stuttgart.de/institut/team/Kimmerle-00003/}}
%\noindent
	\vspace{4pt}
\raggedright{\noindent Hannover, \today}%\quad	\quad	\includegraphics[scale=0.16]{Unterschr.png}
\newpage
%	\cvitem{ }{J{\"u}rgen M{\"u}ller, juergen.mueller@math.rwth-aachen.de, +49 (0)241 809 4551,\newline \url{https://www.math.rwth-aachen.de/~Juergen.Mueller/}}
%\par\bigskip
%\newpage
%\raggedright

\section{Overview of previous research work $\&$ research plan}
In my research projects, I use computer algebra systems in order to discover homological and representation theoretic conjectures. I try to prove or disprove these conjectures with the assistance of various computer algebra systems. This approach is promising, since it is possible to do many detailed calculations with the computer which improves the understanding of and the insight into a given mathematical problem. In the following, a selection of projects I plan to work on is presented. All occurring algebras are finite-dimensional algebras over a field.
\noindent
	\pagestyle{empty}
\pagestyle{fancy}
\fancyhf{}
\fancyhead[L]{\addressfont\itshape\textcolor{gray}{Research statement}}\fancyhead[C]{}\fancyhead[R]{\addressfont\itshape\textcolor{gray}{Bernhard B\"ohmler}}
\fancyfoot[L]{}\fancyfoot[C]{}\fancyfoot[R]{\addressfont\itshape\textcolor{gray}{Page \thepage\ of \pageref{LastPage}}}
\vspace{11pt}\vspace{0.5pt}
\subsection{Brou{\'e}'s abelian defect group conjecture via $p$-permutation equivalences}
This project is concerned with the modular representation theory of finite groups. Trivial source modules, also known as $p$-permutation modules, arise naturally in this context. In order to do calculations with trivial source modules the ordinary characters of their lifts from positive characteristic~$p$ to characteristic zero are of particular interest. The "trivial source character tables" collect information about the character values of trivial source modules with all possible vertices, as well as those of their Brauer constructions.\newline
\noindent During my doctoral thesis I examined trivial source character tables both theoretically and with the help of computer algebra. The results obtained so far naturally lead me to $p$-permutation equivalences.\newline\newline\vspace{-1pt}\noindent Brou{\'e}'s abelian defect group conjecture (ADGC) has generated a lot of interest in recent years. It predicts a categorical equivalence between a block with abelian defect groups and its Brauer correspondent. When formulated in terms of~$p$-permutation equivalences, Brou{\'e}'s ADGC predicts the following: let $G$ be a finite group and let $p$ be a prime number dividing $|G|$. Suppose $k$ is a large enough field. Assume~$b$ is a block of $kG$ with an abelian defect group $D$. Let $H$ be the normaliser of $D$ in~$G$. Let $c$ be the Brauer correspondent of $b$ in $H$. Then, there exists a $p$-permutation equivalence between $b$ and $c$.\newline
\noindent The aim of this project is to find all $p$-permutation equivalences between two arbitrary blocks $b$ and $c$ having the aforementioned properties. Since the number of such equivalences is finte when $G$ and $p$ are fixed, this is indeed feasable. 
\vspace{11pt}\vspace{2pt}
\subsection{Classification of representation-finite gendo-symmetric algebras}
An algebra $B$ is called \textit{gendo-symmetric}, if it has the form $B = \text{End}_A(A \oplus M)$ for a symmetric algebra $A$ and some~$A$-module $M$. Gendo-symmetric algebras are a generalisation of symmetric algebras and contain many important classes of algebras such as Schur algebras and blocks of category $\mathcal{O}$.\newline\newline
%, we refer for example to \url{https://www.ams.org/journals/tran/2016-368-07/S0002-9947-2015-06504-0/} and \url{https://www.ams.org/journals/tran/2011-363-03/S0002-9947-2010-05177-3/home.html} for more information on gendo-symmetric algebras.\newline
\noindent The goal of this project is to classify all representation-finite gendo-symmetric algebras. This was started in my master's thesis.
The problem is reduced to a Dynkin type classification where some extra exceptional cases appear.\newline
\noindent The strategy to solve the remaining open cases ist threefold: we plan to employ results about translation quivers by Gabriel and Riedtmann, calculate small examples for every family of algebras with GAP, and the theory of universal coverings is again available.
\vspace{11pt}\vspace{2pt}
\subsection{Testing homological conjectures via local algebras}
This project is joint work with Ren{\'e} Marczinzik (University of Bonn). The finitistic dimension conjecture is one of the most important homological conjectures for finite-dimensional algebras. It states that the
supremum of all projective dimensions of modules with finite projective dimension is finite.\newline
A consequence of the finitistic dimension conjecture is Tachikawa's conjecture which states that a finite-dimensional algebra $A$ is selfinjective if and only if $Ext_A^i(D(A),A)=0$ for all $i\geq 1$. The latter conjecture is open even for local algebras. There are many related problems. For example, cosider the following questions.
\begin{enumerate}[label=\arabic*),leftmargin=*]
	\item Is there a non-selfinjective local algebra $A$ with $Ext_A^1(D(A),A)=Ext_A^2(D(A),A)=0$ ?
	
	\item If $A$ is commutative and not selfinjective, can we then have $Ext_A^1(D(A),A)=0$ ?
	
	\item When do there exist $d$-cluster tilting modules for local algebras where $d\geq 2$ ?
	
	\item If $A$ is local and self-injective and $M$ is a non-projective module, do we have $Ext_A^1(M,M) \neq 0$?
\end{enumerate}
We remark that Asashiba and Hoshino treated questions 1 and 2 in the case that the Loewy length is at most $3$ where they have proved that there are no such examples. But it seems that nothing is known when the Loewy length is greater than $3$. For point 3 we remark that by now there is only one local algebra $A$ known with a $d$-cluster tilting module $M$ if~$d \geq 2$, namely the algebra $A:=k\langle x,y\rangle /(x^2, x\cdot y+y^2+y^2\cdot x)$.\newline\newline
\noindent This algebra was found by Jan Geuenich using QPA as an example of a non-selfinjective local algebra  with $Ext^1 (D(A),A) =0$.
Ren{\'e} Marczinzik was then able to construct a $2$-precluster tilting object for $A$ and conjectured that this is even $2$-cluster tilting which was subsequently proved by Oeyvind Solberg using QPA.\newline\newline
\noindent Apart from this example no other local algebras having a $d$-cluster-tilting module are known for $d\geq 2$. We plan to attack those $4$ questions with QPA / MAGMA by generating all local quiver algebras of a given dimension over a small finite field.
%As a second goal, I plan to write a program for every inifinite field (that can be entered in GAP) to decompose a module into indecomposable modules which is the only missing tool in our ARQUIVER program to generalise it to arbitrary fields that can be entered in GAP (for example finite field extensions of the rationals).
%At the moment such a decomposition is only possible over finite fields with QPA.


%\newpage
\raggedright

\pagestyle{empty}
\pagestyle{fancy}
\fancyhf{}
\fancyhead[L]{\addressfont\itshape\textcolor{gray}{Research statement}}\fancyhead[C]{}\fancyhead[R]{\addressfont\itshape\textcolor{gray}{Bernhard B\"ohmler}}
\fancyfoot[L]{}\fancyfoot[C]{}\fancyfoot[R]{\addressfont\itshape\textcolor{gray}{Page \thepage\ of \pageref{LastPage}}}













\begin{comment}
\subsection{Testing homological conjectures via local algebras}
This is joint work with Ren{\'e} Marczinzik.
The finitistic dimension conjecture is one the most important homological conjectures for finite-dimensional algebras. It states that the
supremum of all projective dimensions of modules with finite projective dimension is finite.
A consequence of the finitistic dimension conjecture is Tachikawa's conjecture which states that a finite-dimensional algebra $A$ is selfinjective if and only if $Ext_A^i(D(A),A)=0$ for all $i\geq 1$.

The latter conjecture is open even for local algebras. There are many related problems:
\begin{enumerate}
	\item Is there a non-selfinjective local algebra $A$ with $Ext_A^1(D(A),A)=Ext_A^2(D(A),A)=0$ ?
	
	\item If A is commutative and not selfinjective, can we then have $Ext_A^1(D(A),A)=0$ ?
	
	\item When do there exist d-cluster tilting modules for local algebras for $d>=2$ ?
	
	\item If A is local and self-injective and M is a non-projective module, do we have $Ext_A^1(M,M) \neq 0$?
\end{enumerate}
We remark that Asashiba and Hoshino treated questions 1 and 2 in the case that the Loewy length is at most $3$ where they have proved that there are no such examples. But it seems that nothing is known when the Loewy length is greater than $3$. For point 3 we remark that by now there is only one local algebra $A$ known with a $d$-cluster tilting module $M$, if $d \geq 2$, namely the algebra $A:=k\langle x,y\rangle /(x^2, x\cdot y+y^2+y^2\cdot x)$.\newline\newline
\noindent This algebra was found by Jan Geuenich using QPA as an example of a non-selfinjective local algebra  with $Ext^1 (D(A),A) =0$.
Ren{\'e} Marczinzik was then able to construct a $2$-precluster tilting object for $A$ and conjectured that this is even $2$-cluster tilting. This conjecture was proven by Oeyvind Solberg using QPA.\newline\newline
\noindent Apart from this example no other local algebras having a $d$-cluster-tilting module are known for $d\geq 2$.\newline
\noindent We plan to attack those $4$ questions with QPA by trying to generate many admissible ideals generated by polynomials using a given set of monomials and coefficients.
\end{comment}





% DAS STEHT SCHON IM CV !!!!!!! und da ist es die aktuellere Version !!!
%\section{Contact details of referees}
%Prof. Dr. Wolfgang Rump:\newline
%\noindent phone: +49 711 685 65516, e-mail: rump@mathematik.uni-stuttgart.de, University of Stuttgart\newline
%\noindent He was the supervisor of my Master's thesis.\newline\newline\newline
%\noindent Jun.-Prof. Dr. Caroline Lassueur:\newline
%\noindent phone: +49 631 205 2515, e-mail: lassueur@mathematik.uni-kl.de, Technical University of Kaiserslautern\newline
%\noindent She is the supervisor of my doctoral thesis.
%\ \newline
%\ 

%\section{Preferred coworkers}
%to do


\end{document}
